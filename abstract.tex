\documentclass[]{article}
\usepackage{lmodern}
\usepackage{amssymb,amsmath}
\usepackage{ifxetex,ifluatex}
\usepackage{fixltx2e} % provides \textsubscript
\ifnum 0\ifxetex 1\fi\ifluatex 1\fi=0 % if pdftex
  \usepackage[T1]{fontenc}
  \usepackage[utf8]{inputenc}
\else % if luatex or xelatex
  \ifxetex
    \usepackage{mathspec}
  \else
    \usepackage{fontspec}
  \fi
  \defaultfontfeatures{Ligatures=TeX,Scale=MatchLowercase}
\fi
% use upquote if available, for straight quotes in verbatim environments
\IfFileExists{upquote.sty}{\usepackage{upquote}}{}
% use microtype if available
\IfFileExists{microtype.sty}{%
\usepackage{microtype}
\UseMicrotypeSet[protrusion]{basicmath} % disable protrusion for tt fonts
}{}
\usepackage[margin=1in]{geometry}
\usepackage{hyperref}
\hypersetup{unicode=true,
            pdftitle={Abstract for IMPACT@SHU 10th May 2018},
            pdfauthor={Martin Callaghan},
            pdfborder={0 0 0},
            breaklinks=true}
\urlstyle{same}  % don't use monospace font for urls
\usepackage{graphicx,grffile}
\makeatletter
\def\maxwidth{\ifdim\Gin@nat@width>\linewidth\linewidth\else\Gin@nat@width\fi}
\def\maxheight{\ifdim\Gin@nat@height>\textheight\textheight\else\Gin@nat@height\fi}
\makeatother
% Scale images if necessary, so that they will not overflow the page
% margins by default, and it is still possible to overwrite the defaults
% using explicit options in \includegraphics[width, height, ...]{}
\setkeys{Gin}{width=\maxwidth,height=\maxheight,keepaspectratio}
\IfFileExists{parskip.sty}{%
\usepackage{parskip}
}{% else
\setlength{\parindent}{0pt}
\setlength{\parskip}{6pt plus 2pt minus 1pt}
}
\setlength{\emergencystretch}{3em}  % prevent overfull lines
\providecommand{\tightlist}{%
  \setlength{\itemsep}{0pt}\setlength{\parskip}{0pt}}
\setcounter{secnumdepth}{0}
% Redefines (sub)paragraphs to behave more like sections
\ifx\paragraph\undefined\else
\let\oldparagraph\paragraph
\renewcommand{\paragraph}[1]{\oldparagraph{#1}\mbox{}}
\fi
\ifx\subparagraph\undefined\else
\let\oldsubparagraph\subparagraph
\renewcommand{\subparagraph}[1]{\oldsubparagraph{#1}\mbox{}}
\fi

%%% Use protect on footnotes to avoid problems with footnotes in titles
\let\rmarkdownfootnote\footnote%
\def\footnote{\protect\rmarkdownfootnote}

%%% Change title format to be more compact
\usepackage{titling}

% Create subtitle command for use in maketitle
\newcommand{\subtitle}[1]{
  \posttitle{
    \begin{center}\large#1\end{center}
    }
}

\setlength{\droptitle}{-2em}
  \title{Abstract for \href{mailto:IMPACT@SHU}{\nolinkurl{IMPACT@SHU}} 10th May
2018}
  \pretitle{\vspace{\droptitle}\centering\huge}
  \posttitle{\par}
  \author{Martin Callaghan}
  \preauthor{\centering\large\emph}
  \postauthor{\par}
  \predate{\centering\large\emph}
  \postdate{\par}
  \date{09/05/2018}


\begin{document}
\maketitle

\subsection{Abstract}\label{abstract}

Many researchers feel that the only way of demonstrating the impact of
their work is to publish in a reputable journal and then encourage and
measure citations. This is a system that works well for many forms of
research, but there are additional or alternative routes to demonstrate
impact for research activities that involve the development of research
software.

My research involves the investigation, training and use of neural
network models to summarise text at scale and will create a number of
pieces of software. I'm developing code to:

● Clean and format data ● Create and test `Deep Learning' neural
networks ● Train networks against datasets ● Use trained networks to
summarise text ● Measure the effectiveness and efficiency of the
networks and models I have created

This is applied research. I am investigating how to re-use and
re-combine a number of existing tools and techniques in novel ways.

I'm keen to investigate how I will be able to measure the impact of my
code, as this will be my primary output.

In this talk I will discuss, with reference to the code I'm developing:

● The nature and type of research that has a primary output of software
(or `code'). ● How literate programming techniques can help make code
more intelligible and useful to other researchers. ● Tools and methods
to make code discoverable and how preprint servers can help to publicise
this. ● The role of social networks and social reference managers to
help drive impact and exposure of research code. ● Whether funders and
the wider research community understand and value research code and
applied research in the same way that other forms of research output are
valued.


\end{document}
